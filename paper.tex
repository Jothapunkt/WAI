% This is LLNCS.DEM the demonstration file of
% the LaTeX macro package from Springer-Verlag
% for Lecture Notes in Computer Science,
% version 2.4 for LaTeX2e as of 16. April 2010
%
\documentclass{llncs}
%
\usepackage{makeidx} % allows for indexgeneration
\usepackage{graphicx} % to use \includegraphics{...}
\usepackage[utf8]{inputenc}
%
\setcounter{tocdepth}{2}
%
\begin{document}
%
\frontmatter % for the preliminaries
%
\pagestyle{headings}  % switches on printing of running heads
%
\mainmatter % start of the contributions
%
\title{Comparing the Performance of Hash Tables and Balanced Binary Search Trees}
%
%\titlerunning{Comparing Hash Tables and Balanced BST} % abbreviated title (for running head)
%                                      also used for the TOC unless
%                                      \toctitle is used
%
\author{Thomas Dunajski \and Jakob Höfner}

%%%% list of authors for the TOC (use if author list has to be modified)
%\tocauthor{Thomas Dunajski, Jakob Höfner}
%

\institute{Hochschule Darmstadt, Darmstadt, Germany}

\maketitle % typeset the title of the contribution

%%
%% The abstract
%%
\begin{abstract}
Hash Tables are a very widely used data structure that maps keys to values. Using a hash function to compute indices allows them to complete operations very quickly. Balanced Binary Search Trees are trees with branches that are close to equal length, keeping their height to a minimum. Since the cost of most tree operations scales with the trees height, this is very time effective. This paper is dedicated to comparing the perfomance of both data structures. Less experienced programmers might not have any insight into the inner workings of the data structures they are relying on, or where the strengths of the individual structures lie. We performed benchmarks to measure how long it takes to retrieve values depending on the collection size. We also measured how long it takes to retrieve multiple sequential values. These benchmarks showed us that the Hash Table performance is not heavily impacted by the collection size, making it ideal for large collection sized.  The Search Tree was slower when retrieving single entries, but outperformed the Hash Table when multiple sequential values had to be found.
\keywords{hash tables, binary search trees}
\end{abstract}

\tableofcontents
\newpage

%%
%% Include all content.
%%
\section{Introduction}
Associative Arrays or Maps are among the most fundamental data structures. They map keys to values, not allowing duplicate keys.\\
For example, a large data set of employees can be organized by the employee's name. In this case, the name would be the key and an object containing their address, department and other information the value. Using an array or a linked list and iterating over it when retrieving data can be very slow when used on a large set of data, making it necessary to use more elaborate data structures.\\\\
There are two major approaches to implementing a map, namely Search Trees and Hash Tables.
Since these are so commonly used to solve the same problem, in this paper, we want to compare their performances when retrieving values.\\
We want to create benchmarks for different collection sizes and also test the case of retrieving multiple consecutive values when using maps similar to arrays.\\\\
Since these data structures are often already implemented in programming languages, less experienced programmers might not know about the differences between them and what strengths set them apart.
We want to be able to show when to use which by the end of this paper.\\
\subsection{Short Introduction to Hash Tables}
A hash tables is a data structure commonly used for handling large amounts of data.
There are implementations of hash tables in the standard libraries of most major programming languages, e.g. Java HashMaps or Python dictionaries.\\\\
Like other implementations of maps, it stores pairs of unique keys and values they correspond to. The hash tables strength lies in its use of a hash function to map keys to integer indices in a set range. For finding a value, there is no need to iterate over any key-value-pairs, allowing for the operation to be completed in O(1) time[2].\\\\
A good hash function should aim to provide a uniform distribution of all possible keys, meaning there should be few collisions. Collisions can, however, not be avoided entirely, meaning multiple keys will occasionally be mapped to the same index[3].\\\\
To handle these collisions, hash tables do not store values at an index directly. Instead, each index addresses a linked list containing all values whose keys were mapped to that index.
So for retrieving data, the key is first hashed, and then the correct value is found by iterating over the linked list stored at the computed index[2].\\\\
It is possible for all entries in a hash table to be stored in one of these linked lists, in which case finding a value takes O(n) time because one has to iterate over a linked list of length n. This worst case scenario is however highly unlikely when numerous entries are stored in the table[3].

\section{A Short Introduction to Balanced Binary Search Trees}
Binary Search Trees are used to organize data. They are often used in databases and in file systems (Ubiquitous B-Tree). Like Hash Maps, Binary Search Trees store key-value-pairs and do not allow duplicate keys.\\\\
In binary search trees, keys are sorted and placed in a tree structure. Simplified tree nodes, including the root, contain a key-value-pair and up to two nodes. The left subtree only contains keys smaller than the nodes key, the right one only contains greater keys. When finding a key, one starts at the trees root and goes down along its branches depending on whether the key they are looking for is greater or smaller than the key stored at the current node until they reach the node containing that key. (Ubiquitous B-Tree)\\\\
In an unbalanced tree with n keys the worst case scenario for a search is visiting n nodes, when the root and every following node only has one subtree, ending in a single leaf which contains the wanted key. A tree like that is basically structured like a linked list.\\\\
In a Balanced Binary Search Tree, the height of the tree is kept to a minimum. It is possible to structure any tree in a way so that the subtrees of each node have a height difference of at most 1.
This is while still keeping the requirement of the left subtree containing smaller and the right subtree containing greater keys.\\
When searching for a key in a balanced tree, the worst case scenario is visiting log\textsubscript{2}(n) nodes (rounded down). This is when the wanted key is contained in a leaf.\\ While keeping a tree balanced increases search speed it comes at a cost. After each insert and deletion the tree needs to be balanced again using operations such as tree rotations.\\ 
%%
%% The End.
%%
\end{document}
