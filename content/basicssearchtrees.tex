\subsection{Short Introduction to Balanced Binary Search Trees}
This subsection is a short introduction to Balanced Binary Search Trees based on what Douglas Comer said in [1].\\
Binary Search Trees are used to organize data. They are often used in databases and in file systems. Like Hash Maps, Binary Search Trees store key-value-pairs and do not allow duplicate keys.\\\\
In binary search trees, keys are sorted and placed in a tree structure. Simplified tree nodes, including the root, contain a key-value-pair and up to two nodes. The left subtree only contains keys smaller than the nodes key, the right one only contains greater keys. When finding a key, one starts at the trees root and goes down along its branches depending on whether the key they are looking for is greater or smaller than the key stored at the current node until they reach the node containing that key.\\\\
In an unbalanced tree with n keys the worst case scenario for a search is visiting n nodes, when the root and every following node only has one subtree, ending in a single leaf which contains the wanted key. A tree like that is basically structured like a linked list.\\\\
In a Balanced Binary Search Tree, the height of the tree is kept to a minimum. It is possible to structure any tree in a way so that the subtrees of each node have a height difference of at most 1.
This is while still keeping the requirement of the left subtree containing smaller and the right subtree containing greater keys.\\
When searching for a key in a balanced tree, the worst case scenario is visiting log\textsubscript{2}(n) nodes (rounded down). This is when the wanted key is contained in a leaf.\\ While keeping a tree balanced increases search speed it comes at a cost. After each insert and deletion the tree needs to be balanced again.\\ 