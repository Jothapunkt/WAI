\section{Introduction}
Associative Arrays or Maps are among the most fundamental data structures. They map keys to values, not allowing duplicate keys.\\
For example, a large data set of employees can be organized by the employee's name. In this case, the name would be the key and an object containing their address, department and other information the value. Using an array or a linked list and iterating over it when retrieving data can be very slow when used on a large set of data, making it necessary to use more elaborate data structures.\\\\
There are two major approaches to implementing a map, namely Search Trees and Hash Tables.
Since these are so commonly used to solve the same problem, in this paper, we want to compare their performances when retrieving values.\\
We want to create benchmarks for different collection sizes and also test the case of retrieving multiple consecutive values when using maps similar to arrays.\\\\
Since these data structures are often already implemented in programming languages, less experienced programmers might not know about the differences between them and what strengths set them apart.
We want to be able to show when to use which by the end of this paper.\\