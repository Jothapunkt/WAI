\section{A Short Introduction to Hash Tables}
A hash tables is a data structure commonly used for handling large amounts of data.
There are implementations of hash tables in the standard libraries of most major programming languages, e.g. Java, C++, Python and C(?) [Citations for each].\\\\
Like other implementations of maps, it stores pairs of unique keys and values they correspond to. The hash tables strength lies in its use of a hash function to map keys to integer indices in a set range. For finding a value, there is no need to iterate over any key-value-pairs, allowing for the operation to be completed in O(1) time.\\\\
A good hash function should aim to provide a uniform distribution of all possible keys, meaning there should be few collisions. Collisions can, however, not be avoided entirely, meaning multiple keys will occasionally be mapped to the same index.\\\\
To handle these collisions, hash tables do not store values at an index directly. Instead, each index addresses a linked list containing all values whose keys were mapped to that index.
So for retrieving data, the key is first hashed, and then the correct value is found by iterating over the linked list stored at the computed index.\\\\
These linked lists are also known as buckets. It is possible for all entries in a hash table to be stored in one bucket, in which case finding a value takes O(n) time because one has to iterate over a linked list of length n. This worst case scenario is however highly unlikely when numerous entries are stored in the table.
